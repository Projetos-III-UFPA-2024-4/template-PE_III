\textcolor{red}{Leiam com atenção as partes em vermelho!}
\textcolor{red}{ 
\begin{enumerate}
    \item As partes em vermelho fornecem uma breve explicação do modelo de proposta do projeto. 
    \item Os discentes devem remover as partes em vermelho na versão final!
    \item Esse modelo deve ser usado como uma \textbf{ferramenta de planejamento}, portanto, faz parte do projeto de vocês. Planos podem sofrer modificações durante a execução do projeto, contudo uma boa execução depende de um bom plano prévio ao início das atividades de execução do projeto. \textbf{Um bom projeto começa com um bom planejamento}!
    \item Em cada parte, o grupo pode ter flexibilidade para escrever de maneira diferente do que  está sugerido, mas combine com o professor.
\end{enumerate}
}

\subsection{Título do Projeto}

\textcolor{red}{Por exemplo: Projeto de radar de velocidade de carros de baixo custo para ambientes privados.} 

\subsection{Profissionais e entidade(s) participantes}

\textcolor{red}{Relação dos nomes dos participantes do projeto. Incluam o nome e matrícula. Sugestão: procurem também incluir possíveis laboratórios envolvidos, empresas e demais ambientes de execução do projeto.}

\textcolor{red}{\textit{Por exemplo, Fulana (matrícula XXXXXX), Cicrano (matrícula YYYYYYY) e  Beltrano (matríicula ZZZZZZ) serão responsáveis pela execução do projeto sob a tutoria do prof. QQQQQ. Parte do desenvolvimento será realizado no laboratório XXXX, na empresa YYY e nas acomodações de VVVVVVV.}}